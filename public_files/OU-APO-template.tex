%%%%%%%%%%%%%%%%%%%%%%%%%%%%%%%%%%%%%%%%%%%%%%%%%%%%%%%%%%%%%%%%%%%%%%%%
%.PURPOSE        LaTeX template form for OU APO proposals, including
%				 the scientific justification and cover page.
%.Edit History   November 11, 2013 - Jamie Lomax, created file  
%                     Oct 2016 -- John Wisniewski made minor edits
%
%-----------------------------------------------------------------------
%
%
%             OU APO PROPOSAL LATEX TEMPLATE
%             ------------------------------
% The use of this template for the APO proposal scientific justification  
% and cover sheet. Please fill in all required lines.
%
\documentclass[11pt]{article}            
%
\textheight=247mm
\textwidth=180mm
\topmargin=-7mm
\oddsidemargin=-10mm
\evensidemargin=-10mm
\parindent 10pt
%
\begin{document}
%
%
%
%---- APO Cover Page Directions --------------------------------------------
%
% Fill in all the information below for the APO cover sheet. Please note the %following: 
%1) IMPORTANT NOTE - PLEASE READ!: Any material included on the above
%   form when it is submitted for scheduling WILL APPEAR VERBATIM ON
%   A PUBLICALLY ACCESSIBLE WEB SITE.  If you wish for your TAC to
%   consider information that you want kept confidential or restricted
%   in any way, it should be submitted to them separately (from the
%   above form) or you should arrange to have it removed before
%   submission for scheduling.  If you feel it is important for the
%   3.5-meter Director and/or APO staff to also have access to such
%   confidential/restricted information, you must make special and
%   separate arrangements with them; simply identifying such material
%   on the template scheduling request will be ineffective.
%
%2) List all observers.  Remote observing may only be undertaken by,
%   or with the direct help/supervision of, observers with on-site
%   experience and training.  Normally, this is taken to be at least
%   3 nights of time at APO.  
%
%3) For programs carried out remotely, list all observers who are *not*
%   certified for remote operations and state plans for the participation
%   of certified remote observers for all remote observing.  For programs
%   which will be carried out on-site, list all observers who are
%   untrained/inexperienced and state plans for providing the necessary
%   supervision and instruction.  See point #2 immediately above for
%   further details.
%
%4) List all project scientific collaborators and include their
%   institutional affiliation if not from an ARC institution.
%
%5) Indicate whether the time you request is bright, grey or dark or some
%   mixture.  Dark is moon below the horizon; grey is moon up but less than
%   50% phase, and bright is moon up and greater than 50% phase.  It is helpful
%   if you indicate the least restrictive (most moon) conditions which you
%   can use without serious impact on your data.  If omitted, you will probably
%   be given whatever fits most conveniently into the schedule, probably bright
%   time.
%
%6) Telescope time will be scheduled in half night blocks (split at APO
%   solar midnight) for most programs, and time should be requested in
%   these units in most cases.  Scheduling of smaller blocks of time
%   is also routinely accommodated when there is a scientific need.
%   Such programs should request time in units of hours and should clearly
%   state the need for smaller blocks under the OBSERVING SCHEDULE
%   CONSTRAINTS section.  IN EITHER CASE, THE UNIT (HALF NIGHTS OR HOURS)
%   SHOULD BE EXPLICITLY INDICATED.
%
%7) Scheduled science operations must sometimes be canceled for engineering
%   or other purposes.  In some cases Observatory management has limited
%   discretion in the scheduling of such closures.  If there are any reasons
%   that a program deserves special or unusual protection (which, of course,
%   is not always possible) from such interruptions, please state them clearly
%   in the "special protection justification" section.
%
%---- End APO Cover Page Directions -----------------------------------------

%
\section*{\centering{APO Cover Sheet}}
\bigskip

\noindent\textbf{INSTITUTIONAL ID NUMBER:}%leave this blank
\smallskip\\
\textbf{DESCRIPTIVE TITLE:} Astrophysics Research %Title goes here.
\smallskip\\
\textbf{PI:} John Tobin %Your name.
\smallskip\\
\textbf{OBSERVER(S):} Nickalas Reynolds and John Tobin %Note: You need a trained observer here.
\smallskip\\
\textbf{UNCERTIFIED/UNTRAINED OBSERVERS:} Nickalas Reynolds, Joseph Choi, James Derkacy, Alek Kosakowski, Matt Clement, Hora Mishra. Burak Dogruel, Patrick Vallely, Sean Bruton, Rajeeb Sharma, Brian Stephensen
w %People who haven't been to APO.
\smallskip\\
\textbf{COLLABORATORS:} None %People from other institutions.
\smallskip\\
\textbf{CONTACT INFORMATION:} 
Nickalas Reynolds
nickreynolds@ou.edu
Nielsen Hall 406
The University of Oklahoma
https://www.nhn.ou.edu/~reynolds/
%PI email and phone
\smallskip\\
\textbf{HALF NIGHTS OR HOURS REQUESTED:} \\%Remember to specify units of half-nights or hours!
Dark- 1 half night\\
Grey- \\
Bright- \\
\smallskip\\
\textbf{INSTRUMENT:} 
Triplespec \\%Instrument here.
PRIMARY DIS GRATING:\\
SECONDARY DIS GRATING (IF REQUIRED):\\
SLITS/FILTERS/ETC NEEDED: 1.1"
\smallskip\\
\textbf{OBSERVING MODE:} On-Site %Remote or on-site?
\smallskip\\
\textbf{OBSERVING SCHEDULE CONSTRAINTS:} Need an A-half night in Q1 %(e.g. need A-half nights between Jan-Feb; B-half nights between June-July, want full nights instead of half nights, must have March 10 because of simultaneous satellite observations, etc.)
\smallskip\\
\textbf{SPECIAL REQUIREMENTS:} Please don't schedule this obs in Jan %(e.g. nights you can't observe)
\smallskip\\
\textbf{SPECIAL PROTECTION JUSTIFICATION:} None %(e.g. you can't bump this observation for engineering, etc. because it is vital that it be done... - note, this is usually left blank.)
\smallskip\\
\textbf{BRIEF SCIENCE JUSTIFICATION:} There are many problems in astrophysics. I will solve them all with the proposed observations.%You should write a brief abstract here.
\smallskip\\
\textbf{FUTURE TIME NEEDED:} %Please relate if you expect to need additional time in future semesters to complete this project here.
\smallskip\\
\textbf{PUBLICATION PLANS:} %Please briefly describe your plans for publishing data obtained from this program, along with an estimate of the time-scale for publication.
\smallskip\\
\textbf{THESIS?:} No %Please indicate whether these observations are related to a graduate student thesis.



\newpage

%------------- Scientific Justification Directions -------------------------
%The science justification need be no more than a page or two. State the scientific background, pertinent references, and justify your present proposal. In the Science Program section state the immediate objectives of your observations: what will be observed, why observe it, and what information will be extracted from those observations. Under Targets and Feasibility provide a justification of the observing time requested (e.g. why you need all of what you've requested), expected signal-to-noise, and information on target visibility (e.g. visible during A-half nights between January and February, RA and Dec, magnitudes, etc).
%------------- End Scientific Justification Directions ---------------------

\section*{\centering{Scientific Justification}}

Background and justification, what is to be observed and why, what information will be derived, target list, feasibility, signal to noise, etc.

 
%
\smallskip
\noindent\textbf{References}
\smallskip\\

References go here.

\end{document}
%
% Everything following \end{document} is discarded. 
%%%%%%%%%%%%%%%%%%%%%%%%%%%%%%%%%%%%%%%%%%%%%%%%%%%%%%%%%%%%%%%%%%%%%%%%